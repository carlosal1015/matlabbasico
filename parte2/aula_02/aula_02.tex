\documentclass[hyperref={pdfpagelabels=false}]{beamer}
\usetheme[block=fill]{metropolis}

\usepackage[portuguese]{babel}
\usepackage[utf8]{inputenc} % To use characters such as é without typing é
\usepackage{ctable}
\usepackage{listings}
\lstset{%
  language=Matlab,
  showstringspaces=false,
  basicstyle=\linespread{0.9}\ttfamily,
  keywordstyle=\textbf,
  commentstyle=\color{gray},
  stringstyle=\color{orange},
  numbers=left,
  numberstyle=\tiny\color{gray},
  stepnumber=1,
  numbersep=10pt,
  columns=fullflexible,
  tabsize=3,
  frame=single,
  frameround=tttt
}
\let\Tiny=\tiny % eliminates compilation errors
\usepackage{fontspec}

\title{Laboratório de Matemática Computacional II}
\subtitle{Aula 2}
\author[M. Weber Mendonça]{Melissa Weber Mendonça\\
Universidade Federal de Santa Catarina}
\date{2011}

\begin{document}
\setmonofont{Inconsolata}

\begin{frame}
  \titlepage
\end{frame}

\begin{frame}{Na aula passada...}
  {\emph{Slicing}}
  \begin{itemize}
  \item {\texttt{A(i,j)}}
  \item {\texttt{A(i,:)}}
  \item {\texttt{A(:,j)}}
  \item {\texttt{A(:,:)}}
  \item {\texttt{A(:)}} \alert{cuidado!}
  \item {\texttt{A(1:2,:)}}
  \item {\texttt{A(1,2:3)}}
  \end{itemize}
\end{frame}

\begin{frame}{Apagando elementos}
  Podemos apagar elementos de matrizes (ou linhas/colunas inteiras) usando a seguinte sintaxe:
  
  \begin{center} {\texttt{A(i,:) = []}} \end{center}
  \begin{center} {\texttt{A(:,j) = []}} \end{center}
\end{frame}

\begin{frame}{Multiplicação Matriz-Vetor}
  
  $$B = A*v$$
  
  \begin{equation*}
    \begin{pmatrix}
		  b_1\\
		  \alert{b_2}\\
		  \vdots\\
		  b_m
    \end{pmatrix}
    =
    \begin{pmatrix} 
      a_{11} & a_{12} & \cdots & a_{1n}\\ 
      \alert{a_{21}} & \alert{a_{22}} & \alert{\cdots} &
      \alert{a_{2n}}\\
      & & \ddots & \\ 
      a_{m1} & a_{m2} & \cdots & a_{mn} 
    \end{pmatrix}
    \begin{pmatrix}
      \alert{v_1}\\
      \alert{v_2}\\
      \alert{\vdots}\\
      \alert{v_n}
    \end{pmatrix}
  \end{equation*}
  \vfill
  $$b_i = \sum_{k=1}^n a_{ik}v_k$$
  \begin{center} \href{listings/matrizvetor.m}{\underline{\texttt{matrizvetor.m}}} \end{center}
\end{frame}

\begin{frame}{Multiplicação Matriz-Matriz: Elemento a elemento}
  
  $$C = A*B$$

  \footnotesize{%
    \begin{equation*}
      \begin{pmatrix}
        c_{11} & c_{12} & \cdots & c_{1n} \\
        \alert{c_{21}} & c_{22} & \cdots & c_{2n} \\
        &        & \ddots &        \\
        c_{n1} & c_{n2} & \cdots & c_{nn}
      \end{pmatrix}
      =
      \begin{pmatrix} 
        a_{11} & a_{12} & \cdots & a_{1n}\\ 
        \alert{a_{21}} & \alert{a_{22}} & \alert{\cdots} & \alert{a_{2n}}\\
        & & \ddots & \\ 
        a_{n1} & a_{n2} & \cdots & a_{nn} 
      \end{pmatrix}
      \begin{pmatrix}
        \alert{b_{11}} & b_{12} & \cdots & b_{1n}\\ 
        \alert{b_{21}} & b_{22} & \cdots & b_{2n}\\
        \alert{\vdots} & & \ddots & \\ 
        \alert{b_{n1}} & b_{n2} & \cdots & b_{nn} 
      \end{pmatrix}
    \end{equation*}
    }
  \vfill
  $$C_{ij} = \sum_{k=1}^n A_{ik}B_{kj},$$
  
  $A$, $B$ e $C$ $n\times n$. 
  
  \begin{center} \href{listings/matrizmatriz.m}{\underline{\texttt{matrizmatriz.m}}} \end{center}
\end{frame}

\begin{frame}{Multiplicação Matriz-Matriz: Linha $\times$ Coluna}
	
  $$C = A*B$$

  \footnotesize{%
    \begin{equation*}
      \begin{pmatrix}
        c_{11} & c_{12} & \cdots & c_{1n} \\
        \alert{c_{21}} & c_{22} & \cdots & c_{2n} \\
        &        & \ddots &        \\
        c_{n1} & c_{n2} & \cdots & c_{nn}
      \end{pmatrix}
      =
      \begin{pmatrix} 
        \relbar & a(1,:) & \relbar\\
        \alert{\relbar} & \alert{a(2,:)} & \alert{\relbar}\\
        & \vdots & \\
        \relbar & a(n,:) & \relbar 
      \end{pmatrix}
      \begin{pmatrix}
        \alert{|} & | & & |\\ 
        \alert{b(:,1)} & b(:,2) & \cdots & b(:,n)\\
        \alert{|} & | & & | 
      \end{pmatrix}
    \end{equation*}
    }
  \vfill
  $$C(i,j) = A(i,:)*B(:,j)$$
  \begin{center} \href{listings/linhacoluna.m}{\underline{\texttt{linhacoluna.m}}} \end{center}
\end{frame}

\begin{frame}{Multiplicação Matriz-Matriz: Matriz $\times$ Coluna}
  $$C = A*B$$

  \footnotesize{%
    \begin{equation*}
      \begin{pmatrix}
        \alert{|} & & | \\
        \alert{c(:,1)} & \cdots & c(:,n) \\
        \alert{|} & & |
      \end{pmatrix}
      =
      \begin{pmatrix} 
        a_{11} & a_{12} & \cdots & a_{1n}\\ 
        a_{21} & a_{22} & \cdots & a_{2n}\\
        & & \ddots & \\ 
        a_{n1} & a_{n2} & \cdots & a_{nn} 
      \end{pmatrix}
      \begin{pmatrix}
        \alert{|} & & |\\ 
        \alert{b(:,1)} & \cdots & b(:,n)\\
        \alert{|} & & |
      \end{pmatrix}
    \end{equation*}
  }
  \vfill
  $$C(:,j) = A*B(:,j)$$
  \begin{center} \href{listings/matrizcoluna.m}{\underline{\texttt{matrizcoluna.m}}} \end{center}
\end{frame}

\begin{frame}{Multiplicação Matriz-Matriz: Linha $\times$ Matriz}
  $$C = A*B$$
  
  \begin{equation*}
    \begin{pmatrix}
      \alert{\relbar} & \alert{c(1,:)} & \alert{\relbar}\\
      \relbar & c(2,:) & \relbar\\
      &  \vdots      & \\
      \relbar & c(n,:) & \relbar
    \end{pmatrix}
    =
    \begin{pmatrix} 
      \alert{\relbar} & \alert{a(1,:)} & \alert{\relbar}\\ 
      \relbar & a(2,:) & \relbar\\
      & \vdots & \\ 
      \relbar & a(n,:) & \relbar 
    \end{pmatrix}
    \begin{pmatrix}
      b_{11} & b_{12} & \cdots & b_{1n}\\ 
      b_{21} & b_{22} & \cdots & b_{2n}\\
      \vdots & & \ddots & \\ 
      b_{n1} & b_{n2} & \cdots & b_{nn} 
    \end{pmatrix}
  \end{equation*}
  \vfill
  $$C(i,:) = A(i,:)*B$$
  \begin{center} \href{listings/linhamatriz.m}{\underline{\texttt{linhamatriz.m}}} \end{center}
\end{frame}

\begin{frame}{Exercício}
  Dado um vetor com $n$ elementos, construa uma matriz $n\times n$ cuja diagonal principal é esse vetor, e cujos outros elementos são nulos.
  \vfill
  \begin{center} \href{listings/construirdiagonal.m}{\underline{\texttt{construirdiagonal.m}}} \end{center}
\end{frame}

\begin{frame}{Exercício}
  Escreva um programa que tome a diagonal de uma matriz e coloque em um vetor.
  \vfill
  \begin{center} \href{listings/vetordiagonal.m}{\underline{\texttt{vetordiagonal.m}}} \end{center}
  \begin{center} \texttt{v = diag(A)} \end{center}
\end{frame}

\begin{frame}{\texttt{diag(A)}}
  \begin{itemize}
  \item {\texttt{diag(A)}}
  \item {\texttt{diag(v)}}
  \item {\texttt{diag(v,1)}}
  \item {\texttt{diag(A,1)}}
  \item {\texttt{diag(A,-1)}}
  \end{itemize}
\end{frame}

\begin{frame}{Exercício}
  Escreva um programa que calcule o \emph{traço} de uma matriz.
  \vfill
  \begin{center} \href{listings/traco.m}{\underline{\texttt{traco.m}}} \end{center}
  \begin{center}{\texttt{trace(A)}}\end{center}
\end{frame}

\begin{frame}{Exercício}
  Escreva um programa que toma uma matriz diagonal e realiza seu produto com uma matriz qualquer (de forma econômica!)
  \vfill
  \begin{center} \href{listings/diagmatriz.m}{\underline{\texttt{diagmatriz.m}}} \end{center}
\end{frame}

\begin{frame}{Exercício}
  Escreva um programa que realiza o produto entre uma matriz triangular superior e um vetor.
  \vfill
  \begin{center} \href{listings/triuvetor.m}{\underline{\texttt{triuvetor.m}}} \end{center}
\end{frame}

\begin{frame}{Exercício}
  Escreva um programa que realiza o produto entre duas matrizes triangulares inferiores.
  \vfill
  \begin{center} \href{listings/triltril.m}{\underline{\texttt{triltril.m}}} \end{center}
\end{frame}
\end{document}
