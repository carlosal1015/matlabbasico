\documentclass[hyperref={pdfpagelabels=false}]{beamer}
\usetheme[block=fill]{metropolis}

\usepackage[portuguese]{babel}
\usepackage[utf8]{inputenc} % To use characters such as é without typing é
\usepackage{ctable}
\usepackage{listings}
\lstset{%
  language=Matlab,
  showstringspaces=false,
  basicstyle=\linespread{0.9}\ttfamily,
  keywordstyle=\textbf,
  commentstyle=\color{gray},
  stringstyle=\color{orange},
  numbers=left,
  numberstyle=\tiny\color{gray},
  stepnumber=1,
  numbersep=10pt,
  columns=fullflexible,
  tabsize=3,
  frame=single,
  frameround=tttt
}
\let\Tiny=\tiny % eliminates compilation errors
\usepackage{fontspec}

\title{Laboratório de Matemática Computacional II}
\subtitle{Aula 1}
\author[M. Weber Mendonça]{Melissa Weber Mendonça\\
Universidade Federal de Santa Catarina}
\date{2011}

\begin{document}
\setmonofont{Inconsolata}

\begin{frame}
  \titlepage
\end{frame}

\begin{frame}{No semestre passado...}
  \begin{itemize}
  \item Estruturas básicas de programação 
    \begin{itemize}
		\item \emph{if}
		\item \emph{for}
		\item \emph{while}
		\item \emph{e}/\emph{ou}
		\end{itemize}
	\item Estruturas de dados
		\begin{itemize}
		\item Números reais
		\item Vetores (listas)
		\item Matrizes
		\item Texto (strings)
		\end{itemize}
  \end{itemize}
\end{frame}

\begin{frame}{Objetivos neste semestre}
  \begin{itemize}
  \item Mais MATLAB: 
    \begin{itemize}
		\item Matrizes, matrizes, matrizes...
		\item Sistemas lineares
		\item Métodos de álgebra linear
		\item Leitura/escrita em arquivos
		\item Gráficos
		\end{itemize}
  \item Introdução ao \LaTeX 
  \end{itemize}
\end{frame}

\begin{frame}{O que é o \LaTeX?}
  Ferramenta de criação de documentos que permite a formatação correta de fórmulas matemáticas (e desta apresentação).

  Exemplo: \href{listings/testelatex.tex}{\underline{\texttt{testelatex.tex}}} resulta em \href{listings/testelatex.pdf}{\underline{\texttt{testelatex.pdf}}}
\end{frame}

\begin{frame}
  \begin{center} MATLAB \end{center}
\end{frame}

\begin{frame}{Aquecimento I}
  Escreva um programa que recebe dois números inteiros $m$ e $n$, e cria uma matriz com $m$ linhas e $n$ colunas de forma que cada elemento da matriz é igual à soma do seu índice de linha e do seu índice de coluna.
	\vfill
	Exemplo: $a_{12} = 1+2=3$.
  \vfill
  \begin{center} \href{listings/matrizsoma.m}{\underline{\texttt{matrizsoma.m}}} \end{center}
\end{frame}

\begin{frame}{Aquecimento II}
   Uma matriz é triangular superior quando todos os elementos abaixo da sua diagonal principal forem nulos.
	 \vfill
	 Exemplo: \begin{math}
		 \begin{pmatrix}
	     1 & 2 & 1 & 0\\
	     0 & 3 & -1 & 2\\
	     0 & 0 & 2 & 5\\
	     0 & 0 & 0 & 1
	   \end{pmatrix}
	 \end{math}
	 é triangular superior.
	 \vfill
	 Escreva um programa que recebe uma matriz qualquer e a transforma em
   triangular superior.
   \vfill
   \begin{center} \href{listings/triansup.m}{\underline{\texttt{triansup.m}}} \end{center}
\end{frame}

\begin{frame}{Exercício}
  Escreva um programa que identifica se uma matriz é simétrica.
  \vfill
  \begin{center} \href{listings/ehsimetrica.m}{\underline{\texttt{ehsimetrica.m}}} \end{center}
\end{frame}

\begin{frame}{Exercício}
	Uma matriz $A$ de ordem $m\times n$ de números reais pode ser:
	\begin{itemize}
	\item Quadrada, se $m=n$
	  \begin{itemize}
		\item Inversível, $|det(A)|\geq 1e-10$
			\begin{itemize}
			\item Diagonal, se todos os elementos fora da diagonal principal
        forem nulos;
			\item Regular, se não for Diagonal.
			\end{itemize}
		\item Singular, se $|det(A)| < 1e-10$
			\begin{itemize}
			\item Esparsa, se mais da metade das entradas forem nulas;
			\item Densa, se menos da metade das entradas forem nulas.
			\end{itemize}
	  \end{itemize}
	\item Retangular, se $m\ne n$
		\begin{itemize}
		\item "Horizontal", se tiver mais colunas do que linhas
		\item "Vertical", se tiver mais linhas do que colunas.
		\end{itemize}
	\end{itemize}
  Escreva um programa que toma uma matriz qualquer e classifica esta matriz de acordo com as regras acima.
  \vfill
  \begin{center} \href{listings/classificarmatriz.m}{\underline{\texttt{classificamatriz.m}}} \end{center}
\end{frame}

\begin{frame}{Slicing}
  Relembrando: para vetores, 
  \begin{itemize}
	\item {\texttt{v(i)}}
  \item {\texttt{v(:)}}
  \item {\texttt{v(2:4)}}
  \end{itemize}
\end{frame}

\begin{frame}{Slicing, para matrizes}
	Aqui, temos dois índices: o de linhas e o de colunas. Portanto, podemos fazer o slicing nos dois índices. (2 dimensões)
	\begin{itemize}
	\item {\texttt{A(i,j)}}
	\item {\texttt{A(i,:)}}
	\item {\texttt{A(:,j)}}
	\item {\texttt{A(:,:)}}
	\item {\texttt{A(:)}} \alert{cuidado!}
	\item {\texttt{A(1:2,:)}}
	\item {\texttt{A(1,2:3)}}
	\end{itemize}
\end{frame}

\begin{frame}{Exercício}
  Escreva um programa que, usando slicing, substitua uma coluna da matriz por zeros.
  \vfill
  \begin{center} \href{listings/zerarcoluna.m}{\underline{\texttt{zerarcoluna.m}}} \end{center}
\end{frame}

\begin{frame}{Exercício}
  Escreva um programa que, usando slicing, substitua uma linha da matriz por zeros.
  \vfill
  \begin{center} \href{listings/zerarlinha.m}{\underline{\texttt{zerarlinha.m}}} \end{center}
\end{frame}

\begin{frame}{Exercício}
	Escreva um programa que, usando slicing, receba uma matriz $A$ de ordem $m\times n$ e dois números $a$ e $b$ e retorne uma outra matriz formada pelas linha 1 até $a$ e pelas colunas 1 até $b$ de $A$.
  \vfill
  \begin{center} \href{listings/cortarmatriz.m}{\underline{\texttt{cortarmatriz.m}}} \end{center}
\end{frame}

\begin{frame}{Exercício}
	Escreva um programa que, usando slicing, calcule a transposta de uma matriz.
  \vfill
  \begin{center} \href{listings/transpormatriz.m}{\underline{\texttt{transpormatriz.m}}} \end{center}
\end{frame}

\end{document}
