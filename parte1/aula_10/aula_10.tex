\documentclass[hyperref={pdfpagelabels=false}]{beamer}
\usetheme[block=fill]{metropolis}

\usepackage[portuguese]{babel}
\usepackage[utf8]{inputenc} % To use characters such as é without typing é
\usepackage{ctable}
\usepackage{listings}
\lstset{%
  language=Matlab,
  showstringspaces=false,
  basicstyle=\linespread{0.9}\ttfamily,
  keywordstyle=\textbf,
  commentstyle=\color{gray},
  stringstyle=\color{orange},
  numbers=left,
  numberstyle=\tiny\color{gray},
  stepnumber=1,
  numbersep=10pt,
  columns=fullflexible,
  tabsize=3,
  frame=single,
  frameround=tttt}
\let\Tiny=\tiny % eliminates compilation errors
\usepackage{fontspec}

\title{Laboratório de Matemática Computacional I}
\subtitle{Aula 10}
\author[M. Weber Mendonça]{Melissa Weber Mendonça\\
Universidade Federal de Santa Catarina}
\date{2011}

\begin{document}
\setmonofont{Inconsolata}

\begin{frame}
  \titlepage
\end{frame}

\begin{frame}{Exemplo da aula passada}
	Escreva um programa que organiza as entradas de uma lista em ordem crescente.

  \lstinputlisting[title=\texttt{sorting.m}]{listings/sorting.m}
\end{frame}

\begin{frame}{Exercícios}
  Dado o vetor
  \begin{center} {\texttt{vetor = [3,5,4,2,7,1,9,8,6]}} \end{center}
  escreva um programa que toma um valor entre 1 e 9 e retorna sua posição (o índice correspondente) no vetor acima.
  
	Exemplo: vetor(2) = 5. Logo, se a entrada for 5, a saída será 2.
\end{frame}

\begin{frame}{Exemplos}
	\only<1>{Escreva um programa que testa se um determinado número está ou não em uma lista.}
	\only<2>{\begin{center}\lstinputlisting[basicstyle=\linespread{0.9}\small\ttfamily, title=\texttt{estaounao.m}]{listings/estaounao.m}\end{center}}
\end{frame}

\begin{frame}{Exemplos}
	Escreva um programa que encontra todas as ocorrências de um determinado número em uma lista.
  \only<2>{\begin{center}\lstinputlisting[title=\texttt{ocorrencias.m}]{listings/ocorrencias.m}\end{center}}
\end{frame}

\begin{frame}{Exemplos}
  Escreva um programa que recebe uma lista e 2 índices $i$ e $j$, e troca o elemento de índice $i$ pelo elemento de índice $j$.
	\begin{center} \lstinputlisting[title=\texttt{trocaelemento.m}]{listings/trocaelemento.m} \end{center}
\end{frame}
\end{document}
