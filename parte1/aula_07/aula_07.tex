\documentclass[hyperref={pdfpagelabels=false}]{beamer}
\usetheme[block=fill]{metropolis}

\usepackage[portuguese]{babel}
\usepackage[utf8]{inputenc} % To use characters such as é without typing é
\usepackage{ctable}
\usepackage{listings}
\lstset{%
  language=Matlab,
  showstringspaces=false,
  basicstyle=\linespread{0.9}\ttfamily,
  keywordstyle=\textbf,
  commentstyle=\color{gray},
  stringstyle=\color{orange},
  numbers=left,
  numberstyle=\tiny\color{gray},
  stepnumber=1,
  numbersep=10pt,
  columns=fullflexible,
  tabsize=3,
  frame=single,
  frameround=tttt}
\let\Tiny=\tiny % eliminates compilation errors
\usepackage{fontspec}

\title{Laboratório de Matemática Computacional I}
\subtitle{Aula 7}
\author[M. Weber Mendonça]{Melissa Weber Mendonça\\
Universidade Federal de Santa Catarina}
\date{2011}

\begin{document}
\setmonofont{Inconsolata}

\begin{frame}
  \titlepage
\end{frame}

\begin{frame}{Aquecimento}
	Escrever uma função que receba dois textos e concatene os dois.
	
	\only<2>{\lstinputlisting[title=\texttt{concatenar.m}]{listings/concatenar.m}}
\end{frame}

\begin{frame}{Funções para texto}
  \begin{itemize} 
  \item {\texttt{[]}} ou {\texttt{strcat(texto1,texto2,...)}}: Concatena
    mais de um texto.
  \item {\texttt{length(texto)}}: Calcula o comprimento de um texto. 
  \end{itemize}
\end{frame}

\begin{frame}{Exemplos}
	Escrever uma função que, dado um texto, adiciona um ponto final neste texto.
  
 	\only<2>{\lstinputlisting[title=\texttt{ponto.m}]{listings/ponto.m}}
\end{frame}

\begin{frame}{Exemplos} 
  Escrever um programa que, dado um número inteiro $n$, imprime $n$ pontos na tela.
  
  Exemplo:

  \qquad Entrada: 5

  \qquad Saída: {\texttt{>> .....}}
  
  \only<2>{\lstinputlisting[title=\texttt{pontinhos.m}]{listings/pontinhos.m}}
\end{frame}

\begin{frame}{Exemplos}
  Dados dois números $a$ e $b$ entre 1 e 20, desenhar na tela o retângulo de área $ab$.
  
	\only<2>{\lstinputlisting[title=\texttt{retangulo.m}]{listings/retangulo.m}}
\end{frame}

\begin{frame}{Indices}
   Um texto funciona por índices: 
   \begin{itemize}
   \item \texttt{texto = 'Melissa'}
   \item \texttt{texto(1) = 'M'}
   \item \texttt{texto(2) = 'e'}
   \item \texttt{texto(1:2) = 'Me'}
   \end{itemize}
\end{frame}

\begin{frame}{Exemplos}
  Dado um texto, eliminar o primeiro e o último caracteres deste texto.
  
  \only<2>{\lstinputlisting[title=\texttt{primeiroeultimo.m}]{listings/primeiroeultimo.m}}
\end{frame}

\begin{frame}{Exemplo}
  Escreva um programa que leia uma frase e imprima esta frase na vertical (em pé).

	Exemplo:

  \textbf{Entrada:} 'Frase'. \textbf{Saída:} \begin{tabular}{c} F\\r\\a\\s\\e\end{tabular}

	\only<2>{\lstinputlisting[title=\texttt{vertical.m}]{listings/vertical.m}}
\end{frame}

\begin{frame}{Exemplos}
  Dada uma frase, composta por várias palavras separadas por espaços, contar o número de espaços entre as palavras.
	\begin{center} \href{listings/espacos.m}{\underline{\texttt{espacos.m}}} \end{center}
\end{frame}

\begin{frame}{Exemplos}
	Escrever uma função que recebe como entradas um texto e um caracter (letra), e conta quantas ocorrências desta letra existem no texto.

	\only<2>{\lstinputlisting[title=\texttt{quantas.m}]{listings/quantas.m}}
\end{frame}

\end{document}
