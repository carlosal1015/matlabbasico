\documentclass[hyperref={pdfpagelabels=false}]{beamer}
\usetheme[block=fill]{metropolis}

\usepackage[portuguese]{babel}
\usepackage[utf8]{inputenc} % To use characters such as é without typing é
\usepackage{ctable}
\usepackage{listings}
\lstset{%
  language=Matlab,
  showstringspaces=false,
  basicstyle=\linespread{0.9}\ttfamily,
  keywordstyle=\textbf,
  commentstyle=\color{gray},
  stringstyle=\color{orange},
  numbers=left,
  numberstyle=\tiny\color{gray},
  stepnumber=1,
  numbersep=10pt,
  columns=fullflexible,
  tabsize=3,
  frame=single,
  frameround=tttt}
\let\Tiny=\tiny % eliminates compilation errors
\usepackage{fontspec}

\title{Laboratório de Matemática Computacional I}
\subtitle{Aula 5}
\author[M. Weber Mendonça]{Melissa Weber Mendonça\\
Universidade Federal de Santa Catarina}
\date{2011}

\begin{document}
\setmonofont{Inconsolata}

\begin{frame}
  \titlepage
\end{frame}

\begin{frame}{Comentários}
  No MATLAB, os comentários são sinalizados por \%
  \only<2>{\lstinputlisting[title=\texttt{leitor2.m}]{listings/leitor2.m}}
\end{frame}

\begin{frame}{For}
  Escreva um programa que imprima todos os números de 1 a $n$, onde $n$ é dado pelo usuário.
  \begin{itemize}
  \item[] {\texttt{n = input('Entre com o numero n: ')}}
  \item[] {\texttt{numero = 1;}}
  \item[] {\texttt{\alert{Enquanto} numero $\leq n$}}
    \begin{itemize}
    \item[] {\texttt{numero}}
    \item[] {\texttt{numero = numero+1;}}
    \end{itemize}
  \item[] {\texttt{\alert{Fim Enquanto}}}
  \end{itemize}
\end{frame}

\begin{frame}{For}
  Escreva um programa que imprima todos os números de 1 a $n$, onde $n$ é dado pelo usuário.
  \begin{itemize}
  \item[] {\texttt{n = input('Entre com o numero n: ')}}
  \item[] {\texttt{\alert{Para} numero = 1 \alert{até} numero = n}}
    \begin{itemize}
    \item[] {\texttt{numero}}
    \end{itemize}
  \item[] {\texttt{\alert{Fim Para}}}
  \end{itemize}
\end{frame}

\begin{frame}{For}
	Escreva um programa que imprima todos os números de 1 a $n$, onde $n$ é dado pelo usuário.
	\lstinputlisting[title=\texttt{for1.m}]{listings/for1.m}
\end{frame}

\begin{frame}{Exemplos}
  Escrever um programa que calcule o fatorial de um número, usando {\texttt{for}}.
  \vfill
  \begin{center} \href{listings/fatfor.m}{\underline{\texttt{fatfor.m}}} \end{center}
\end{frame}

\begin{frame}{Exemplos}
  Somar todos os números naturais de 0 até 1000 que sejam múltiplos de 3 ou 5.
  \vfill
  \begin{center} \href{listings/tresoucinco.m}{\underline{\texttt{tresoucinco.m}}} \end{center}
\end{frame}

\begin{frame}{Exemplos}
  Escrever um programa que imprime os números naturais entre 0 e $n$, trocando os múltiplos de 5 por \alert{plim!}
  \vfill
  \begin{center} \href{listings/plim.m}{\underline{\texttt{plim.m}}} \end{center}
\end{frame}

\begin{frame}{Exemplos}
  Escrever um programa que multiplica dois números (dados pelo usuário) através de \emph{somas}.
   
  Exemplo: $3\times 5 = 5+5+5$.
  \vfill
  \begin{center} \href{listings/multiplicacao.m}{\underline{\texttt{multiplicacao.m}}} \end{center}
\end{frame}

\begin{frame}{Exemplos}
	A sequência de Fibonacci é dada pela regra seguinte: dados $F_0 = 0$ e $F_1 = 1$, o $n$-ésimo termo é calculado pela fórmula
	$$F_n = F_{n-1} + F_{n-2}.$$
	Escrever um programa que calcula o $n$-ésimo termo da sequência de Fibonacci.
	\vfill
  \begin{center} \href{listings/fibonacci.m}{\underline{\texttt{fibonacci.m}}} \end{center}
\end{frame}

\begin{frame}{Exemplos}
	Escrever um programa que aproxima $\pi$, usando a fórmula
	$$\pi \approx 4\sum_{k=0}^m \frac{(-1)^k}{2k+1},$$
	sendo que $m$ é informado pelo usuário.
	\vfill
  \begin{center} \href{listings/piaproximado.m}{\underline{\texttt{piaproximado.m}}} \end{center}
\end{frame}

\begin{frame}{Funções}
  Na matemática, 
	$$f(x) = y.$$
	
	Entrada: $x$
	
	Saída: $y$
	
	Ação: $f$.
	
	Exemplo: $f(x) = x^2$.
\end{frame}

\begin{frame}{Funções pré-definidas}
  \begin{itemize}
    \item \texttt{n = \textbf{input}('Entre com um numero:')}
    \item \texttt{nfat = \textbf{factorial}(n)}
    \item \texttt{texto = \textbf{num2str}(25)}
  \end{itemize}
\end{frame}

\begin{frame}{Funções - No Matlab}

  \begin{itemize}
  \item[] \texttt{function \alert{saída} = \alert{nome}(\alert{entrada})}
	\item[] \begin{itemize}
	\item[] \texttt{\alert{comandos}}
	\end{itemize}
	\end{itemize}
	As funções devem estar num arquivo próprio, cujo nome deve ser igual ao nome da função.
\end{frame}

\begin{frame}{Qual a diferença entre um \emph{script} e uma \emph{função}?}
  Um \emph{script} é um arquivo que contém uma sequência de comandos, mas não exige entrada ou saída.
  
  Uma função deve, obrigatoriamente, ter pelo menos uma entrada e uma saída.
\end{frame}

\begin{frame}{Exemplo: script}
  \lstinputlisting[title=\texttt{testemelissa.m}]{listings/testemelissa.m}
\end{frame}

\begin{frame}{Exemplo: função}
  \lstinputlisting[title=\texttt{xquadrado.m}]{listings/xquadrado.m}
  \vfill
	Note que não usamos {\texttt{clear}} dentro da função, pois dentro da função as variáveis são \alert{locais}.
\end{frame}

\begin{frame}{Argumentos de entrada e saída}
  Se tivermos mais de um argumento de entrada, basta separá-los por vírgulas:
  
  \texttt{s = soma(x,y)}
  
  Se tivermos mais de um argumento de saída, precisamos escrevê-los entre colchetes:
  
  \texttt{[a,b] = somaesubtracao(x,y)}
\end{frame}

\begin{frame}{Exemplos}
  \begin{itemize}
  \item[] \texttt{function s = adicionar(x,y)}
  \item[] \begin{itemize}
  \item[] \texttt{s = x+y}
  \end{itemize}
  \end{itemize}
\end{frame}

\begin{frame}{Exemplos}
   Escrever uma função que, dada a base e a altura de um triângulo, calcula sua área.
   
	 \only<2>{ \vfill
     \begin{itemize}
     \item[] \texttt{function area = areadotriangulo(b,h)}
     \item[] \begin{itemize}
     \item[] \texttt{area = b*h/2;}
     \end{itemize}
     \end{itemize}
   }
\end{frame}

\begin{frame}{Exemplos}
	Escrever uma função que recebe uma temperatura em Celsius e converte para Fahrenheit, sendo que a fórmula de conversão é
	$$T_f = \frac{9 T_c}{5}+32$$
	\only<2>{\lstinputlisting[title=\texttt{temperatura.m}]{listings/temperatura.m}}
\end{frame}

\end{document}
