\documentclass[hyperref={pdfpagelabels=false}]{beamer}
\usetheme[block=fill]{metropolis}

\usepackage[portuguese]{babel}
\usepackage[utf8]{inputenc} % To use characters such as é without typing é
\usepackage{ctable}
\usepackage{listings}
\lstset{%
  language=Matlab,
  showstringspaces=false,
  basicstyle=\linespread{0.9}\ttfamily,
  keywordstyle=\textbf,
  commentstyle=\color{red},
  stringstyle=\color{orange},
  numbers=left,
  numberstyle=\tiny\color{gray},
  stepnumber=1,
  numbersep=10pt,
  columns=fullflexible,
  tabsize=3,
  frame=single,
  frameround=tttt}
\let\Tiny=\tiny % eliminates compilation errors
\usepackage{fontspec}

\title{Laboratório de Matemática Computacional I}
\subtitle{Aula 3}
\author[M. Weber Mendonça]{Melissa Weber Mendonça\\
Universidade Federal de Santa Catarina}
\date{2011}

\begin{document}
\setmonofont{Inconsolata}

\begin{frame}
  \titlepage
\end{frame}

\begin{frame}{Na aula passada...}
  \begin{itemize}
  \item {\texttt{if - (elseif) - else - end}}
  \item {\texttt{\&\&}}      
  \end{itemize}
\end{frame}

\begin{frame}{Para lembrar:}
  Escrever um programa para ler três números e escrever se estes podem ou não formar um triângulo. Para formar os lados de um triângulo cada um dos valores tem que ser menor que a soma dos outros dois.
\end{frame}

\begin{frame}{Resposta:}
  \begin{center}
    \begin{minipage}{0.95\textwidth}
      \lstinputlisting[title=\texttt{triangulo.m}]{listings/triangulo.m}
    \end{minipage}
  \end{center}
\end{frame}

\begin{frame}{Exercício}
  Escrever um programa que diga se um número é natural, inteiro ou real. ({\texttt{floor}})
  \begin{center}
    \begin{minipage}{0.8\textwidth}
      \only<2>{\lstinputlisting[title=\texttt{tipodenum2.m}]{listings/tipodenum2.m}}
      \only<3>{\lstinputlisting[title=\texttt{tipodenum.m}]{listings/tipodenum.m}}
    \end{minipage}
  \end{center}
\end{frame}

\begin{frame}{Exercício}
  \only<1>{Escrever um programa que decida se uma pessoa é criança, adolescente, adulto ou idoso \alert{e que mostre o resultado conforme o gênero da pessoa.}}
  \only<2>{\lstinputlisting[title=\texttt{pessoa.m}]{listings/pessoa.m}}
  \only<3>{\begin{center}\href{listings/tipodepessoa.m}{\underline{\texttt{tipodepessoa.m}}}\end{center}}
  \only<4>{\begin{center}\href{listings/tipodepessoa2.m}{\underline{\texttt{tipodepessoa2.m}}}\end{center}}
  \only<5>{\begin{center}\href{listings/tipodepessoa3.m}{\underline{\texttt{tipodepessoa3.m}}}\end{center}}
\end{frame}

\begin{frame}{ou - \texttt{$||$}}
  Escrever um algoritmo que decide se um carro precisa da primeira revisão.

  \vfill 
  \only<2>{
    \begin{itemize}
    \item[] Se (idade do carro $\geq$ 1 ano)
      \begin{itemize}
      \item[] Faça a revisão.
      \end{itemize}
    \item[] Fim Se
    \item[] Se (quilometragem do carro $\geq$ 10.000 km)
      \begin{itemize}
      \item[] Faça a revisão.
      \end{itemize}
    \item[] Fim Se
    \end{itemize}
  }
  \only<3>{
    \begin{itemize}
    \item[] Se (idade do carro $\geq$ 1 ano) \alert{ou} (quilometragem do carro
      $\geq$ 10.000 km)
      \begin{itemize}
      \item[] Faça a revisão.
      \end{itemize}
    \item[] Fim Se
    \end{itemize}
  }

  \only<2-3>{
    \begin{alertblock}{}
      \begin{center}
        Se qualquer uma das duas condições for satisfeita,\\ tomaremos a mesma decisão.
      \end{center}
    \end{alertblock}
    }
\end{frame}

\begin{frame}{ou - {\texttt{$||$}}: Tabela Verdade}
  \setbeamercovered{invisible}
  \begin{center}
    \begin{tabular}{c c c c}
      \texttt{a} & \texttt{b} & & \texttt{a $||$ b}\\\toprule
      \only<1->{Falso} & \only<1->{Falso} & & \only<2->{Falso}\\\midrule
      \only<3->{Verdadeiro} & \only<3->{Falso} & &\only<4->{Verdadeiro}\\\midrule
      \only<5->{Falso} & \only<5->{Verdadeiro} & &\only<6->{Verdadeiro}\\\midrule
      \only<7->{Verdadeiro} & \only<7->{Verdadeiro} & &\only<8->{Verdadeiro}\\\bottomrule
    \end{tabular}
  \end{center}
\end{frame}

\begin{frame}{ou - {\texttt{$||$}}: Tabela Verdade (Matlab)}
  \begin{center}
    \begin{tabular}{c c c c}
      \texttt{a} & \texttt{b} & & \texttt{a $||$ b}\\\toprule
      0 & 0 & & 0\\\midrule
      1 & 0 & &1\\\midrule
      0 & 1 & &1\\\midrule
      1 & 1 & &1\\\bottomrule
    \end{tabular}
  \end{center}
\end{frame}

\begin{frame}{Exercícios}
  \only<1>{Escrever um programa que decida se uma pessoa deve pagar meia entrada em um show.}
  \only<2>{\begin{center} \href{listings/meia.m}{\underline{\texttt{meia.m}}} \end{center}}
  \only<3>{Escrever um programa que, dado um mês, diga quantos dias tem esse mês.}
  \only<4>{\begin{center} \href{listings/diasdomes.m}{\underline{\texttt{diasdomes.m}}} \end{center}}
\end{frame}

\begin{frame}{Voltando ao exemplo da outra aula...}

   Queremos que o usuário tente adivinhar um número entre 0 e 10.

   MAS: gostaríamos que o programa soubesse que, \emph{enquanto a pessoa não acertar o número, ele deve continuar perguntando}.
\end{frame}

\begin{frame}{Voltando ao exemplo do número...}
  \begin{itemize}
  \item[] {\texttt{numero = input('Entre com o numero: ')}}
  \item[] {\texttt{\alert{Enquanto} numero $\sim$= 5}}
    \begin{itemize}
    \item[] {\texttt{disp('Errou... Tente novamente!')}}
    \item[] {\texttt{numero = input('Entre com o numero: ')}}
    \end{itemize}
  \item[] {\texttt{\alert{Fim Enquanto}}}
  \item[] {\texttt{disp('Acertou!')}}
  \end{itemize}
\end{frame}

\begin{frame}{Voltando ao exemplo do número...}
  \lstinputlisting[title=\texttt{guesswhile.m}]{listings/guesswhile.m}
\end{frame}

\begin{frame}{While-End}
  A estrutura {\texttt{while}} é uma estrutura de \alert{repetição} que serve para repetir um bloco de código até que certa condição seja satisfeita.

  No caso anterior, repetimos as linhas 3 e 4 do código até que a variável \textbf{numero} contenha o valor 5.

  \begin{itemize}
  \item[] {\texttt{while (condição)}}
    \begin{itemize}
    \item[] {\texttt{faça ...}}
    \end{itemize}
  \item[] {\texttt{end}}
  \end{itemize}
\end{frame}

\begin{frame}{Exemplo}
  Escrever na tela os números de 1 a 10.
  
  \only<3>{\lstinputlisting[title=\texttt{while.m}]{listings/while.m}}
  \only<4>{\lstinputlisting[title=\texttt{while2.m}]{listings/while2.m}}
\end{frame}

\begin{frame}{Exemplo}
  Escrever um programa que imprime 'Repetindo!' na tela até que o usuário digite o número 0.
\end{frame}

\begin{frame}{Exemplos extra}
  \begin{itemize}
  \item \href{listings/guess.m}{\underline{\texttt{guess.m}}}
  \item \href{listings/guess2.m}{\underline{\texttt{guess2.m}}}
  \item \href{listings/fat.m}{\underline{\texttt{fat.m}}}
  \item \href{listings/leitor.m}{\underline{\texttt{leitor.m}}}
  \item \href{listings/media.m}{\underline{\texttt{media.m}}}
  \item \href{listings/numprimo.m}{\underline{\texttt{numprimo.m}}}
  \end{itemize}
\end{frame}

\end{document}
