\documentclass[hyperref={pdfpagelabels=false}]{beamer}
\usetheme[block=fill]{metropolis}

\usepackage[portuguese]{babel}
\usepackage[utf8]{inputenc} % To use characters such as é without typing é
\usepackage{ctable}
\usepackage{listings}
\lstset{%
  language=Matlab,
  showstringspaces=false,
  basicstyle=\linespread{0.9}\ttfamily,
  keywordstyle=\textbf,
  commentstyle=\color{gray},
  stringstyle=\color{orange},
  numbers=left,
  numberstyle=\tiny\color{gray},
  stepnumber=1,
  numbersep=10pt,
  columns=fullflexible,
  tabsize=3,
  frame=single,
  frameround=tttt}
\let\Tiny=\tiny % eliminates compilation errors
\usepackage{fontspec}

\title{Laboratório de Matemática Computacional I}
\subtitle{Aula 9}
\author[M. Weber Mendonça]{Melissa Weber Mendonça\\
Universidade Federal de Santa Catarina}
\date{2011}

\begin{document}
\setmonofont{Inconsolata}

\begin{frame}
  \titlepage
\end{frame}

\begin{frame}{Exercícios}
  Dado um texto, retornar o mesmo texto ao contrário.

  Exemplo: Entrada: \texttt{'umdoistres'} Saída: \texttt{'sertsiodmu'}

	\lstinputlisting[title=\texttt{contrario.m}]{listings/contrario.m}
\end{frame}

\begin{frame}{Exercícios}
	O que este código faz?
	\lstinputlisting[title=\texttt{oquefaz.m}]{listings/oquefaz.m}
\end{frame}

\begin{frame}{Exercícios}
  O que este código faz?
  \lstinputlisting[title=\texttt{oquefaz2.m}]{listings/oquefaz2.m}
\end{frame}

\begin{frame}{Exercícios}
	\only<1>{Encontre o(s) erro(s) neste código:
	  \lstinputlisting[title=\texttt{erro.m}]{listings/erro.m}}
	\only<2>{\lstinputlisting[title=\texttt{errocorrigido.m}]{listings/errocorrigido.m}}
\end{frame}

\begin{frame}{Listas}
  Já vimos que um texto é uma sequência de caracteres (letras, espaços, pontos, etc...)
	\begin{center}
		{\texttt{texto = 'Um texto eh uma sequencia de caracteres.'}}
	\end{center}
	\begin{itemize}
		\item {\texttt{texto(1) = 'U'}}
		\item {\texttt{texto(length(texto)) = '.'}}
		\item {\texttt{texto(2:5) = 'm te'}}
	\end{itemize}
\end{frame}

\begin{frame}{Listas}
   Podemos também fazer listas de números:
  	\begin{center}{\texttt{lista = [1,3,4,5,7]}}\end{center}
	  \begin{itemize}
		\item {\texttt{lista(1) = 1}}
		\item {\texttt{lista(length(lista)) = 7}}
		\item {\texttt{lista(2:5) = [3,4,5,7]}}
	  \end{itemize}
\end{frame}

\begin{frame}{Exemplos}
	Escreva um programa que peça para o usuário entrar com uma lista de números, e calcula a média das entradas desta lista.
	\begin{center} \href{listings/calculamedia.m}{\underline{\texttt{calculamedia.m}}} \end{center}
	\begin{center} \href{listings/calculamedia2.m}{\underline{\texttt{calculamedia2.m}}} \end{center}
\end{frame}

\begin{frame}{Exemplos}
	\only<1>{Escreva um programa que encontra a maior entrada de uma lista.\\
		\begin{center} \href{listings/maiorelemento.m}{\underline{\texttt{maiorelemento.m}}} \end{center}}
	\only<2>{Escreva uma função que encontra a maior entrada de uma lista.\\
    \begin{center} \href{listings/maior.m}{\underline{\texttt{maior.m}}} \end{center}}
\end{frame}

\begin{frame}{Exemplos}
		Escreva um programa que organiza as entradas de uma lista em ordem crescente. (Resposta na próxima aula...)
\end{frame}

\end{document}
