\documentclass[hyperref={pdfpagelabels=false}]{beamer}
\usetheme[block=fill]{metropolis}

\usepackage[portuguese]{babel}
\usepackage[utf8]{inputenc} % To use characters such as é without typing é
\usepackage{ctable}
\usepackage{listings}
\lstset{%
  language=Matlab,
  showstringspaces=false,
  basicstyle=\linespread{0.9}\ttfamily,
  keywordstyle=\textbf,
  commentstyle=\color{gray},
  stringstyle=\color{orange},
  numbers=left,
  numberstyle=\tiny\color{gray},
  stepnumber=1,
  numbersep=10pt,
  columns=fullflexible,
  tabsize=3,
  frame=single,
  frameround=tttt}
\let\Tiny=\tiny % eliminates compilation errors
\usepackage{fontspec}

\title{Laboratório de Matemática Computacional I}
\subtitle{Aula 4}
\author[M. Weber Mendonça]{Melissa Weber Mendonça\\
Universidade Federal de Santa Catarina}
\date{2011}

\begin{document}
\setmonofont{Inconsolata}

\begin{frame}
  \titlepage
\end{frame}

\begin{frame}{Tipos de Variável}

	Até aqui, estamos lidando com dois tipos de variável: números e \emph{texto}.
  
	\begin{alertblock}{}
		\begin{center}
			O valor de uma variável do tipo \emph{texto} sempre será \\escrito
			entre	aspas!
		\end{center}
	\end{alertblock}

  \only<2->{
    \begin{itemize}
    \item {\texttt{str2num}}: transforma um número em texto
    \item {\texttt{num2str}}: transforma texto em número
    \end{itemize}
    \begin{itemize}
    \item[] {\texttt{num2str(10) = '10'}} $\leftarrow$ serve para imprimir um\\
		  \hspace{4.15cm} número no meio de um texto.
    \item[] {\texttt{str2num('25') = 25}} 
    \item[] {\texttt{num2str('texto') = ?}}
    \end{itemize}
  }
\end{frame}

\begin{frame}{Exemplo ({\texttt{num2str}})}
  \lstinputlisting[title=\texttt{dispteste.m}]{listings/dispteste.m}
\end{frame}

\begin{frame}{Exemplo ({\ttfamily{num2str}})}
	\begin{center} \href{listings/dispteste2.m}{\underline{\texttt{dispteste2.m}}} \end{center}
\end{frame}

\begin{frame}{Voltando ao exemplo da outra aula...}

  Queremos que o usuário tente adivinhar um número entre 0 e 10.
  
  MAS: gostaríamos que o programa soubesse que, \emph{enquanto a pessoa não acertar o número, ele deve continuar perguntando}.
\end{frame}

\begin{frame}{Voltando ao exemplo do número...}
  \begin{itemize}
  \item[] {\texttt{numero = input('Entre com o numero: ')}}
  \item[] {\texttt{\alert{Enquanto} numero $\sim$= 5}}
    \begin{itemize}
    \item[] {\texttt{disp('Errou... Tente novamente!')}}
    \item[] {\texttt{numero = input('Entre com o numero: ')}}
    \end{itemize}
  \item[] {\texttt{\alert{Fim Enquanto}}}
  \item[] {\texttt{disp('Acertou!')}}
  \end{itemize}
\end{frame}

\begin{frame}{Voltando ao exemplo do número...}
  \lstinputlisting[title=\texttt{guesswhile.m}]{listings/guesswhile.m}
\end{frame}

\begin{frame}{While-End}
  A estrutura {\texttt{while}} é uma estrutura de \alert{repetição} que serve para repetir um bloco de código até que certa condição seja satisfeita.

   No caso anterior, repetimos as linhas 3 e 4 do código até que a variável \texttt{numero} contenha o valor 5.

   \begin{itemize}
   \item[] {\texttt{while (condição)}}
     \begin{itemize}
     \item[] {\texttt{faça ...}}
     \end{itemize}
   \item[] {\texttt{end}}
   \end{itemize}
\end{frame}

\begin{frame}{Exemplo}
  Escrever na tela os números de 1 a 10.
	\lstinputlisting[title=\texttt{while2.m}]{listings/while2.m}
\end{frame}

\begin{frame}{Exemplo}
  Escrever um programa que imprime 'Repetindo!' na tela até que o usuário digite o número 0.
	\vfill
	\begin{center} \href{listings/repetindo.m}{\underline{\texttt{repetindo.m}}} \end{center}
\end{frame}

\begin{frame}{Exemplo}
  Escrever um programa que escreve na tela todos os números pares entre 1 e 100.
	\vfill
	\begin{center} \href{listings/pares.m}{\underline{\texttt{pares.m}}} \end{center}
\end{frame}

\begin{frame}{Exemplo}
  Escrever um programa que calcula a soma dos $n$ primeiros números naturais ($n$ é fornecido pelo usuário).
	\vfill
	\begin{center} \href{listings/somannaturais.m}{\underline{\texttt{somannaturais.m}}} \end{center}
\end{frame}

\begin{frame}{Exemplos}
  Escrever um programa que calcule o fatorial de um número.
  \only<2>{\lstinputlisting[title=\texttt{fat.m}]{listings/fat.m}}
\end{frame}

\begin{frame}{Exemplo}
  Escrever um programa que decide se um número é primo.
  \vfill 
  \begin{center} \href{listings/primo.m}{\underline{\texttt{primo.m}}} \end{center}
\end{frame}

\begin{frame}{Exemplo}
  \only<1>{Escrever um programa que calcula a média de idade de um grupo de pessoas. O número de pessoas é determinado pelo usuário, que deve entrar as idades uma por uma. Quando o usuário digitar o número 0, a lista estará completa.}
  \only<2>{\lstinputlisting[title=\texttt{media.m}]{listings/media.m}}
\end{frame}
\end{document}
