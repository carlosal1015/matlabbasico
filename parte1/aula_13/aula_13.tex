\documentclass[hyperref={pdfpagelabels=false}]{beamer}
\usetheme[block=fill]{metropolis}

\usepackage[portuguese]{babel}
\usepackage[utf8]{inputenc} % To use characters such as é without typing é
\usepackage{ctable}
\usepackage{listings}
\lstset{%
  language=Matlab,
  showstringspaces=false,
  basicstyle=\linespread{0.9}\ttfamily,
  keywordstyle=\textbf,
  commentstyle=\color{gray},
  stringstyle=\color{orange},
  numbers=left,
  numberstyle=\tiny\color{gray},
  stepnumber=1,
  numbersep=10pt,
  columns=fullflexible,
  tabsize=3,
  frame=single,
  frameround=tttt}
\let\Tiny=\tiny % eliminates compilation errors
\usepackage{fontspec}

\title{Laboratório de Matemática Computacional I}
\subtitle{Aula 13}
\author[M. Weber Mendonça]{Melissa Weber Mendonça\\
Universidade Federal de Santa Catarina}
\date{2011}

\begin{document}
\setmonofont{Inconsolata}

\begin{frame}
  \titlepage
\end{frame}

\begin{frame}{Aquecimento}
  Escrever um programa que soma duas matrizes.
  \vfill
  \begin{center} \href{listings/somarmatrizes.m}{\underline{\texttt{somarmatrizes.m}}} \end{center}
\end{frame}

\begin{frame}{Matrizes Especiais}
  \begin{center}
    {\texttt{eye(m,n)}}\\
    {\texttt{zeros(m,n)}}\\
    {\texttt{ones(m,n)}}
  \end{center}
\end{frame}

\begin{frame}{Transposição}
  Do mesmo jeito que fizemos para vetores, podemos verificar as dimensões de uma matriz pelo comando\\
	\begin{center} {\texttt{size(A)}} \end{center}
	Além disso, a transposta $A^T$ da matriz $A$ pode ser calculada com
	\begin{center} {\texttt{A'}} ou {\texttt{transpose(A)}}\end{center}
\end{frame}

\begin{frame}{Exemplo}
  Escrever um programa que calcula o determinante de uma matriz 2$\times $2.
  \vfill
  \begin{center} \href{listings/determinante.m}{\underline{\texttt{determinante.m}}} \end{center}
\end{frame}

\begin{frame}{Exemplo}
  Escreva uma função que recebe uma matriz $A$ e um vetor $v$, e:
  \begin{itemize}
  \item[(i)] Calcula um vetor coluna $u$ onde cada elemento é a soma de todos os elementos de cada linha da matriz $A$;
  \item[(ii)] Retorna $u-v$.
  \end{itemize}
  \vfill
  \begin{center} \href{listings/funcaomatriz.m}{\underline{\texttt{funcaomatriz.m}}} \end{center}
\end{frame}

\begin{frame}{Exercício}
  Escreva um programa que encontre o maior elemento de uma matriz e informe sua localização.
   \vfill 
   \begin{center} \href{listings/maiormatriz.m}{\underline{\texttt{maiormatriz.m}}} \end{center}
\end{frame}

\begin{frame}{Exercício}
  Escreva um programa que constrói a matriz que representa o tabuleiro de xadrez, onde 1 representa uma casa preta, e 0 representa uma casa branca.
	\vfill
	\begin{center} \href{listings/xadrez.m}{\underline{\texttt{xadrez.m}}} \end{center}
\end{frame}

\end{document}
