\documentclass[hyperref={pdfpagelabels=false}]{beamer}
\usetheme[block=fill]{metropolis}

\usepackage[portuguese]{babel}
\usepackage[utf8]{inputenc} % To use characters such as é without typing é
\usepackage{ctable}
\usepackage{listings}
\lstset{%
  language=Matlab,
  showstringspaces=false,
  basicstyle=\linespread{0.9}\ttfamily,
  keywordstyle=\textbf,
  commentstyle=\color{gray},
  stringstyle=\color{orange},
  numbers=left,
  numberstyle=\tiny\color{gray},
  stepnumber=1,
  numbersep=10pt,
  columns=fullflexible,
  tabsize=3,
  frame=single,
  frameround=tttt}
\let\Tiny=\tiny % eliminates compilation errors
\usepackage{fontspec}

\title{Laboratório de Matemática Computacional I}
\subtitle{Aula 11}
\author[M. Weber Mendonça]{Melissa Weber Mendonça\\
Universidade Federal de Santa Catarina}
\date{2011}

\begin{document}
\setmonofont{Inconsolata}

\begin{frame}
  \titlepage
\end{frame}

\begin{frame}{Aquecimento}
	Escreva um programa que conta quantas vezes uma vogal aparece em um texto e armazena a quantidade em uma lista, onde o primeiro elemento é a quantidade de ``a'', o segundo elemento é a quantidade de ``e'', e assim por diante.

	Exemplo: ``Este eh um texto.''
	
	Resposta: [0, 4, 0, 1, 1]
	
	\begin{center} \href{listings/contavogais.m}{\underline{\texttt{contavogais.m}}} \end{center}
\end{frame}

\begin{frame}{Append}
  Em uma lista, podemos acrescentar elementos a qualquer momento:
  \vfill
  Exemplo:\\
  
  \hskip2cm {\texttt{lista = [1,3,4,5]}}\\
  \hskip2cm {\texttt{lista = [lista 2]}}\\
  \hskip2cm {\texttt{lista}}
\end{frame}

\begin{frame}{Exemplos}
  Podemos também efetuar operações com listas: adição, subtração, multiplicação por escalar, divisão...

	\begin{block}{}
    \begin{center}
      Testar operaçoes com listas e escalares!
    \end{center}
  \end{block}
\end{frame}

\begin{frame}{Exemplos}
	Escreva um programa que imprime na tela a tabuada de um número $n$.
	
	\begin{center} \href{listings/tabuada.m}{\underline{\texttt{tabuada.m}}} \end{center}
\end{frame}

\begin{frame}{Exercício}
		Escrever um programa que recebe duas listas de números, listando as dezenas e as unidades de certos números, e retorna uma lista com os números completos.
    \vfill
    Exemplo:	
 	  \begin{tabular}{l r c l}
 		  Entrada:  & dezenas &=& [2,1,0,7,0,0,3]\\
      & unidades &=& [3,2,4,6,1,8,4]\\
      Saída:   & lista &=& [23,12,4,76,1,8,34]
 	  \end{tabular}

	  \begin{center} \href{listings/juntanumeros.m}{\underline{\texttt{juntanumeros.m}}} \end{center}
\end{frame}

\begin{frame}{Exercício}
  Escreva um programa que recebe os coeficientes de um polinômio (números positivos), e escreve o polinômio na forma padrão. 
  \vfill
	Exemplo: coeficientes = [1,2,3,4,5]
	
	polinômio: {\texttt{p(x) = 1 + 2x + 3x\^{}2 + 4x\^{}3 + 5x\^{}4}}
	
	\begin{center} \href{listings/polinomio.m}{\underline{\texttt{polinomio.m}}} \end{center}
\end{frame}

\begin{frame}\frametitle{Orientação}
   Quando fazemos um vetor, podemos escolher entre um vetor ``na horizontal''
   e um vetor ``na vertical'':

	\begin{center}
		{\texttt{vetor = [1,2,3,4,5]}}\\
		{\texttt{vetor = [1 2 3 4 5]}}\\
		{\texttt{vetor = [1;2;3;4;5]}}
	\end{center}
 
	\begin{alertblock}{Cuidado!}
		Para realizar operações entre vetores, estes devem ter o mesmo tamanho e a mesma orientação!
	\end{alertblock}
	\begin{center} {\texttt{size(vetor)}} \end{center}
\end{frame}

\begin{frame}{Exemplo}
	Escreva um programa que toma um vetor em uma orientação e transforma-o em outra. (Se o vetor for vertical, transformar em horizontal; se for horizontal, transformá-lo em vertical.)
	
	\begin{center} \href{listings/mudarorientacao.m}{\underline{\texttt{mudarorientacao.m}}} \end{center}
	\vfill
	\only<2>{Para alternar entre uma e outra orientação, podemos usar	o operador {\texttt{'}}:
		\begin{center} {\texttt{vetor'}} \end{center}}
\end{frame}

\begin{frame}{Produto Interno}
  O produto (elemento a elemento) de dois vetores é realizado pelo operador {\texttt{*}}. Se quisermos calcular o produto interno de dois vetores, precisamos fazer
  \begin{center} {\texttt{vetor1*vetor2'}} \end{center}
  ou
  \begin{center} {\texttt{vetor1'*vetor2}} \end{center}
\end{frame}
\end{document}
